\documentclass[12pt]{report}
\usepackage[utf8]{inputenc}
\usepackage[OT1]{fontenc} 
\usepackage{xspace}
\usepackage[frenchb]{babel}
\usepackage{times}
\usepackage{graphicx}
\usepackage[top=2in, bottom=1.5in, left=1in, right=1in]{geometry}

\graphicspath{{images/}}
\addtocounter{tocdepth}{3}
\setcounter{secnumdepth}{3}
\bibliographystyle{plain}
 
\begin{document}

\newcommand\university{Université de Grenoble}
\newcommand\school{École doctorale EEATS}
\newcommand\structure{Laboratoire: Inria Grenoble Rhône-Alpes \\ Entreprise: Aldebaran}
\newcommand\auteur{Jory Lafaye}
\newcommand\encadrant{Dr. Pierre-Brice Wieber, Inria\\Dr. Cyrille Collette, Aldebaran\\Dr. Sebastien Dalibard, Aldebaran}
\newcommand\directeur{Dr. Bernard Brogliato, Inria\\~\\~}
\newcommand\titre{Commande des mouvements et de l'équilibre \\d'un robot humanoïde à roues omnidirectionnelles}

\begin{titlepage}

\newcommand{\HRule}{\rule{\linewidth}{0.5mm}} % Defines a new command for the horizontal lines, change thickness here

\center % Center everything on the page
 
%----------------------------------------------------------------------------------------
%	HEADING SECTIONS
%----------------------------------------------------------------------------------------

\textsc{\LARGE \university}\\[0.5cm] 
\textsc{\Large \school}\\[1.5cm] 
\textsc{\Large Thèse CIFRE}\\ 
\textsc{Présentée par}\\[0.5cm] 
\textsc{\Large \auteur}\\ [1.5cm] 
\textsc{\large \structure}\\[0.5cm] 

%----------------------------------------------------------------------------------------
%	TITLE SECTION
%----------------------------------------------------------------------------------------

\HRule \\[0.4cm]
{ \LARGE \bfseries \titre}\\[0.4cm] % Title of your document
\HRule \\[1.5cm]
 
%----------------------------------------------------------------------------------------
%	AUTHOR SECTION
%----------------------------------------------------------------------------------------

\begin{minipage}{0.4\textwidth}
\begin{flushleft} 
\emph{Directeur:}\\
\directeur
\end{flushleft}
\end{minipage}
~
\begin{minipage}{0.4\textwidth}
\begin{flushright} 
\emph{Encadrants:} \\
\encadrant
\end{flushright}
\end{minipage}\\[4cm]

% If you don't want a supervisor, uncomment the two lines below and remove the section above
%\Large \emph{Author:}\\
%John \textsc{Smith}\\[3cm] % Your name

%----------------------------------------------------------------------------------------
%	DATE SECTION
%----------------------------------------------------------------------------------------

%{\large Le \today}\\[3cm] % Date, change the \today to a set date if you want to be precise

%----------------------------------------------------------------------------------------
%	LOGO SECTION
%----------------------------------------------------------------------------------------

%\includegraphics{Logo}\\[1cm] % Include a department/university logo - this will require the graphicx package
 
%----------------------------------------------------------------------------------------

\vfill % Fill the rest of the page with whitespace

\end{titlepage}

\tableofcontents

%
\newpage\cite{bib.miasa.2010}
%


\addcontentsline{toc}{chapter}{Résumé}

\chapter{Introduction}~
	\section{Présentation de la plateforme expérimentale}
		\subsection{Pepper, un robot humanoïde à roues omnidirectionnelles}
		\subsection{Capteurs et actionneurs}
		\subsection{Propriétés mécaniques}	
	\section{État de l'art}
		\subsection{Problématiques associées à Pepper}
		\subsection{Commande et équilibre des robots à roues}
			\subsubsection{Les robots à une et deux roues}
			\subsubsection{Les robots à trois roues et plus}
		\subsection{Commande et équilibre des robots bipèdes}
		\subsection{Synthèse et conclusion}
	\section{Organisation du document}
\chapter{Modélisation et commande de Pepper}~
	\section{Modélisation dynamique}
		\subsection{Choix du modèle et conséquences}
		\subsection{Équations de la dynamique}
		\subsection{Linéarisation et approximations}
	\section{Commande prédictive}
		\subsection{Modélisation de la dynamique future}
		\subsection{Formulation du problème d'optimisation}
			\subsubsection{Choix du type d'optimisation}
			\subsubsection{Formulation des objectifs}
			\subsubsection{Formulation des contraintes}
	\section{Méthode de résolution du problème}
		\subsection{Principe de la programmation quadratique}
		\subsection{Application à la commande prédictive}
			\subsubsection{Linéarisation des contraintes}
			\subsubsection{Formulation mathématique finale}
		\subsection{Implémentation logicielle : ``MPC-WalkGen''}
	\section{Résultats et expérimentations}
		\subsection{Protocole expérimental}
		\subsection{De l'importance du choix des pondérations}
		\subsection{Expérimentations}
		\subsection{Vers un choix automatique des pondérations}
\chapter{Prise en compte du basculement de Pepper}~
	\section{Modélisation dynamique}
		\subsection{Problématique supplémentaire}
		\subsection{Équations de la dynamique}
		\subsection{Linéarisation et approximations}
	\section{Commande prédictive}
		\subsection{Choix du type d'optimisation}
		\subsection{Formulation des objectifs}
		\subsection{Formulation des contraintes}
	\section{Gestion des deux modèles dynamiques exclusifs}
		\subsection{Choix d'un superviseur et conséquences}
		\subsection{Fonctionnement du superviseur}
		\subsection{Fonctionnement de l'estimateur d'impact}
	\section{Résultats et expérimentations}
		\subsection{Protocole expérimental}
		\subsection{Expérimentations}
		\subsection{Limites physiques et algorithmiques}
	\section{Vers une modélisation unifiée des deux dynamiques}
		\subsection{Problème de complémentarité linéaire}
		\subsection{Méthodes de résolution}
			\subsubsection{Programmation quadratique avec contraintes non-linéaire}
			\subsubsection{Linéarisation par \textit{apriori}}
			\subsubsection{Conclusion}
\chapter{Synthèse}
	\section{Contributions}
	\section{Perspectives}
	\section{conclusion}
	
\addcontentsline{toc}{chapter}{Bibliographie}
\bibliography{bib}

\addcontentsline{toc}{chapter}{Annexes}
\appendix
\chapter{Optimisation du choix du modèle dynamique}
\chapter{Résolution d'un problème quadratique}


\end{document}