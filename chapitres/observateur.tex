\chapter{Mesures et observateurs}
	\section{Les différentes valeurs à observer}

		- Position / vitesse / accélération base et corps

		- angle / vitesse angulaire / accélération angulaire basculement base

		- angle de la pente

	\section{Capteurs disponibles}

		- mre / imu

	\section{Méthodes de mesure et conséquences}
		\subsection{Mesure de la posture du robot}

			- En utilisant les mre et le modèle théorique du robot
			
			- Nécessite un bon modèle du robot et une bonne calibration

		\subsection{Observation de la position de la base mobile}
		\label{section.observateurbase}
			- On mesure la vitesse des roues. On en déduit la position de la base en intégrant dans le temps et en utilisant un modèle des roues

			- Dérive due à l'intégration. Ne mesure pas les glissements sur le sol

		\subsection{Observation des vitesses et accélérations du robot et de la base}

			- Celles-ci sont observées en utilisant la prédiction du mouvement du robot au prochain pas de temps (utilisation de la dynamique d'ordre 3)

			- Mieux que dériver la psoition du robot, moins de sensibilité au bruit (dérivation, quantification, capteur)

			- Moins réactif aux erreurs en vitesses et accélérations

			- Le robot étant commencé en position pour le cors, et en vitesse pour les roues, cela n'a pas grande importance

		\subsection{Observation de l'angle de basculement et d'inclinaison du sol}
		\label{section.mesure_tilt_pente}
			
			- Système de base non observable. On mesure la somme des deux angles avec les accéléros, et la somme des variations angulaire avec les gyro.

			- Pas de capteurs de force sur les roues. On ne sait pas lesquelles sont au sol.

			- Il faut faire des hypothèses pour rendre le système observable

			- Problèmes : Non-détections et faux-positifs

			- Considère que lorsque l'angle est constant, alors on est sur une pente

			- Sinon, toute variation de l'angle est considéré comme un push

			- Problèmes si on perturbe le robot avec une dynamique lente

			