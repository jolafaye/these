\section{Modélisation dynamique}
\subsection{Choix du modèle et conséquences}

Parler des modèles n-corps. (Annexe A)
Comparaisons entre ces modèles et le modèle corps complet.
Choix entre complexité (nombre de variables) et fidélité avec le robot réel.
Pertinence du modèle deux corps pour Pepper.

\subsection{Équations de la dynamique}

Détailler les équations de la dynamique.
CoP pour synthétiser les forces de contact.
Contraintes sur le CoP.
Différences entre un modèle deux masses et un modèle une masse (moment pris en compte)

\subsection{Linéarisation et approximations}

Considérer mouvement horizontal.
Négliger les moments de chaque corps (masse point).
Considérer la gravité comme constante


\section{Commande prédictive}
\subsection{Modélisation de la dynamique future}

Choix d'une dynamique polynomiale et de l'ordre de la dynamique.
Détailler les équations.
Ecrire l'équations du CoP


\subsection{Formulation du problème d'optimisation}
\subsubsection{Choix du type d'optimisation}

Choix des variables
minimiser une norme 2 sous contraintes
Choix de résolution du multi objectif par pondération.

\subsubsection{Formulation des objectifs}

Détailler les différents objectifs

\subsubsection{Formulation des contraintes}

Détailler les différentes contraintes (sous la forme non linéaire)

\section{Méthode de résolution du problème}
\subsection{Principe de la programmation quadratique}

Présenter la résolution de problème d'optimisation quadratique sous contrainte linéaire.
Avantages et inconvénients.
(Annexe B)

\subsection{Application à la commande prédictive}
\subsubsection{Linéarisation des contraintes}

Linéariser les contraintes

\subsubsection{Formulation mathématique finale}

Ecrire l'équations.\\
Parler des pondérations.\\
conclure.

\subsection{Schéma de contrôle}

Présentation du schéma de contrôle: feedback en position / commande en vitesse cartésienne et positions articulaires et vitesse des roues.\\
Parler des différents retards.\\
Présentation de la méthode de compensation des retards (extrapolation).\\
Parler de la compensation des jeux mécaniques (dead-zone).\\
Réglage de la stabilité (\% de command-sensor).\\
Vitesse des boucles de contrôle.\\
asservissement bas niveau.

\subsection{Implémentation logicielle : ``MPC-WalkGen''}

Présenter mpc-walkgen

\section{Résultats et expérimentations}
\subsection{Protocole expérimental}

Expériences de trajectoire non réalisable.\\
Choix du sol.\\
Choix des roues.

\subsection{Expérimentations}

Analyser les difféentes expériences

\subsection{Vers un choix automatique des pondérations}

Présenter l'inconvénient des pondérations fixes.\\
Nécessité de méta-paramètres pour régler automatiquement les pondérations.