\chapter{Modélisation du système}
\section{Choix du modèle et conséquences}
	
	- Choix d'un modèle dynamique rigide multi corps
	
	- Compromis fidélité/compléxité et temps de calcul
	
	- Choix du nombre de corps (lien vers anexe pour une optimisation des valeur).
	
	- Phénomènes physiques non-pris en compte : Mécanique de contact roue/sol + jeu articulaire + élasticité hip roll + moments des différents sous corps rigide
	
	\section{Modélisation dynamique}

		\subsection{Problème de complémentarité mixte}
		
		- Présentation des variables (c, b, forces de contact sur chaque roues)
		
		- Equations des énergies cin/pot

		- Contraintes sur la position de b donc problème de comlémentarité sur les forces de contact

		- Problème de résolution de ce problème de complémentarité mixte, il faut donc le séparer en plusieurs parties
		
		\subsection{Les trois roues en contact avec le sol}

		- Etat des forces de contact définies (toutes en contact)

		- En dériver l'équation du cop (barycentre des forces de contact)

		- Linéarisation et approximations
		
		\subsection{Le robot bascule sur deux roues}

		- Etat des forces de contact définies (seul deux des forces sont en contact)

		- Changement de variable pour utiliser l'angle de basculement

		- En dériver l'équation liant l'angle, c et b. 
		
		- Linéarisation et approximations

	\section{Modélisation de la dynamique future}
		\subsection{Nécessité de prédire le futur}

		- Les contraintes dynamiques sont trop fortes pour autoriser un contrôle sans prédiction du futur à haute accélération.

		- Démontrer en calculant les accélérations limites dans différents cas

		- Non nécéssité d'un modèle dynamique précis dans le futur (feedback, on ne calcule que la première commande)

		- Permet d'assurer une stabilité à long terme (quelques secondes)

		\subsection{Choix de la dynamique d'extrapolation}

		- Contraintes : Linéarité entre les variables / accélérations continues donc polynome d'ordre 3

		- Formulation de l'équation d'état

		- Calcul des dérivées

		\subsection{Formulation du modèle prédictif}

		- Formulation du modèle prédictif

		- Problème de controlabilité dans lecas de basculement.

		- Inversions de matrice