\chapter{Commande par modèle prédictif}
\label{chapitre.commande}
	\section{Principe}
	
		L'objectif de ce chapitre est de présenter la loi de commande permettant de réaliser les mouvements et l'équilibre d'un robot humanoïde à roues omnidirectionnelles.
		Dans le chapitre \rf{chapitre.modele}, nous avons développé un modèle du robot utilisant deux corps.
		Les contraintes de complémentarités dues aux forces de contact nous ont emmené à modéliser le robot en deux parties : lorsque celui-ci est sur ces trois roues, et lorsqu'il bascule sur deux de ses roues.
		Ensuite de quoi, nous avons montré qu'il est important de modéliser la dynamique dans le futur à cause des contraintes sur le CoP, qui limitent fortement les mouvements réalisables par le robot.
		Enfin, deux équations prédictives de la dynamique ont été formulées :
		\liste{
			\item Dans le cas où le robot possède trois roues sur le sol, l'équation \rf{eq.dyn_cop_pred} exprime la relation entre la position du CoP et la dynamique des corps du robot.
			\item Dans le cas où le robot bascule sur deux roues, l'équation \rf{eq.dyn_tilt_pred} exprime la dynamique de l'angle de basculement en fonction de celle des corps du robot.
		}
		
		Nous avons donc choisi de réaliser une loi de commande optimale basée sur un modèle prédictif linéaire en les variables de commande, soumis à des contraintes linéaires.
		La fonction objectif de cette commande optimale est basée sur la minimisation d'une norme d'ordre $2$, permettant la convergence d'une grandeur d'erreur vers $0$, utilisés notamment dans une tâche de suivi de trajectoire.
		
		L'interêt de l'aspect prédictif de cette commande est de pouvoir utiliser les connaissances que l'on a sur le comportement dynamique du système ainsi que sur les mouvements demandés, pour prévoir et adapter en avance les trajectoires résultantes
		afin d'obtenir une meilleure réalisation des objectifs dans le temps. 
		Cette adaptation est nécessaire lorsque les contraintes du système réduisent de beaucoup l'ensemble des trajectoires atteignables instantanément par les corps du robot, ce qui est notre cas.
	
	\section{Outil mathématique et contraintes associées}
	
		Notre objectif est de faire tourner la loi de commande en temps réel sur la plateforme expérimentale. 
		Il est donc important d'utiliser un outil mathématique de calcul de la loi de commande qui permette cela.
		De plus, notre système étant fortement contraint (Position du CoP, limites articulaires, limites dynamique des moteurs), nous ne pouvons disposer d'une solution analytique.
		
		Nous avons donc choisi d'utiliser la méthode de la programmation quadratique sous contraintes linéaires.
		Cette méthode de formulation d'un problème d'optimisation permet de minimiser une fonction objectif quadratique en les variables d'optimisation, où celles-ci sont soumises à des contraintes d'inégalités linéaires :
		\eq{
			\lst{
				\min\limits_x \prt{\tr{x}Qx+\tr{p}x} \\
				v^- \leq Vx \leq v^+
			}
		}
		avec $x$ le vecteur des variables d'optomisation, $Q$ la matrice hessienne, $p$ le vecteur linéaire, $V$ la matrice des contrainte et $v^+$ et $v^-$ les vecteurs linéaire des contraintes.
		
		Une fois le problème posé, un solveur de problème quadratique permet, en temps réel, de calculer la solution du problème de minimisation en un certain nombre d'itérations, dépendant des contraintes.
		L’annexe \rf{annexe.qp} détaille les méthodes de résolutions d'un problème quadratique.
		\newpage\newpage
		
		


			- On veut faire tourner le programme rapidement.

			- Il n'existe généralement pas de solution analytique à un problème d'optimisation sous contrainte

			- on ne peut guère aller plus compliqué qu'une résolution quadratique sous contrainte linéaire

			- On va donc utiliser une formulation de QP

			- Ce type d'optimisation nous permet de minimiser une norme 2, ce qui est suffisant. Le temps de calcul ne dépend pas du nombre d'objectifs.

			- Il faudra linéariser les contraintes du problème

			- Le temps de calcul dépend du nombre de contraintes, il faudra donc choisir un ensemble de contraintes linéaires conservatives suffisament petit, mais sans restreindre trop le système.

			- Lien vers l'anexe our expliquer comment on résoud un qp

	\section{Formulation des problèmes d'optimisations}
		\subsection{Introduction}

			- On ne peut pas résoudre simplement un problème de complémentarité mixte

			- On décide de séparer la résolution du problème en 3 parties

			- Expliquer les deux premières, dépendant des dynamiques

			- Expliquer le problème avec la transition, et la non gestion de l'impact.

			- Un superviseur est écrit permettant de gérer les différents états.

		\subsection{Lorsque les trois roues sont en contact avec le sol}
			\label{section.mpc_trois_roues}
			\subsubsection{Formulation des objectifs}
			\label{section.objectifs3roues}

				- Tracking control

				- Robustesse (CoP)

				- Stabilité numérique (jerk)

			\subsubsection{Formulation des contraintes}

				- Respecter la dynamique : CoP

				- Limites vitesses/accélérations de la base

				- Respecter la cinématique : C-B

			\subsubsection{Problème quadratique résultant}
		
				- Ecrire le problème résultant

		\subsection{Lorsque le robot bascule sur deux roues}
			\label{section.mpc_deux_roues}
			\subsubsection{Formulation des objectifs}
		
				- Minimiser l'angle

				- Minimiser la vitesse angulaire

				- Stabilité numérique

			\subsubsection{Formulation des contraintes}

				- Contrainte sur l'angle > 0

				- Respecter la cinématique : C-B

				- Limites vitesses/accélérations de la base

			\subsubsection{Problème quadratique résultant}
		
				- Ecrire le problème résultant

		\subsection{Gestion de la transition entre les deux états}
			\subsubsection{Formulation des objectifs}

				- Minimiser la vitesse

				- Robustesse (CoP)

				- Stabilité numérique (jerk)

			\subsubsection{Formulation des contraintes}

				- Respecter la dynamique : CoP

				- Limites vitesses/accélérations de la base

				- Respecter la cinématique : C-B

			\subsubsection{Problème quadratique résultant}
		
				- Ecrire le problème résultant

	\section{Gestion des deux modèles dynamiques exclusifs}
		\label{section.superviseur}
		\subsection{Choix d'un superviseur et conséquences}

			- Problème de transitions entre les controlleurs

			- Il faut un superviseur qui gère les différents états

			- Parler de l'estimateur d'impact

			- Limitations due au superviseur : Détection tardive / inadéquate / Choix non optimal / Oscillations

			- Avantages : Gérer de manière simple différents modèles dynamiques

		\subsection{Fonctionnement du superviseur}

			- Expliquer le fonctionnement du superviseur et des différents états

		\subsection{Fonctionnement de l'estimateur d'impact}

			- Détailler le fonctionnement de l'estimateur d'impact

	\section{Vers une modélisation unifiée des deux dynamiques}
		\label{section.modelisation_unifiee}
		\subsection{Problème de complémentarité linéaire}

			- Considérer uniquement un problème de basculement dans une direction

			- dire que par la suite, se limiter à ce cas permet de gérer tout les cas, en faisant quelques hyothèses

			- Enoncer la dynamique de complémentarité

			- Problème : Il y a $2^n$ états possibles linéaires à la dynamique.

		\subsection{Méthodes de résolution}
				- Considérer que lorsque le robot ne bascle pas, la commande ne le fera pas basculer. On se retrouver dans le cas du premier programme d'optimisation uniquement

				- Si un basculement est mesure, faire un apriori qu'il n'y aura pas de rebond possible. Ainsi, il n'y a qu'une variable à choisir : le temps d'impact.

				- On se retrouve avec un problème non-linéaire, qui devient linéaire en choisissant l'état de cette variable.

				- Il y a n choix possibles.

				- On peut résoudre n QP et choisir le plus optimal.

				- Ou alors on peut résoudre 3 QP et faire converger l'état de la variable.

				- Présenter le problème d'optimisation unifié