\chapter{Commande par modèle prédictif}
	\section{Principe}

		- Commande optimale sous contraintes

		- Résoudre le problème sur un horizon donné, en utilisant un modèle prédisant le futur

		- La solution optimale du système n'est pas connu si l'on ne connait pas less objectifs et contraintes futures

		- Exemple de faire un déplacement triangle

		- Utilisation de la commande optimale dans la marche bipède.

		- Intéret lorsque les contraintes sont fortes par rapport aux dynamiques de mandées de mouvement

	\section{Outil mathématique et contraintes associées}

			- On veut faire tourner le programme rapidement.

			- Il n'existe généralement pas de solution analytique à un problème d'optimisation sous contrainte

			- on ne peut guère aller plus compliqué qu'une résolution quadratique sous contrainte linéaire

			- On va donc utiliser une formulation de QP

			- Ce type d'optimisation nous permet de minimiser une norme 2, ce qui est suffisant. Le temps de calcul ne dépend pas du nombre d'objectifs.

			- Il faudra linéariser les contraintes du problème

			- Le temps de calcul dépend du nombre de contraintes, il faudra donc choisir un ensemble de contraintes linéaires conservatives suffisament petit, mais sans restreindre trop le système.

			- Lien vers l'anexe our expliquer comment on résoud un qp

	\section{Formulation des problèmes d'optimisations}
		\subsection{Introduction}

			- On ne peut pas résoudre simplement un problème de complémentarité mixte

			- On décide de séparer la résolution du problème en 3 parties

			- Expliquer les deux premières, dépendant des dynamiques

			- Expliquer le problème avec la transition, et la non gestion de l'impact.

			- Un superviseur est écrit permettant de gérer les différents états.

		\subsection{Lorsque les trois roues sont en contact avec le sol}
			\subsubsection{Formulation des objectifs}

				- Tracking control

				- Robustesse (CoP)

				- Stabilité numérique (jerk)

			\subsubsection{Formulation des contraintes}

				- Respecter la dynamique : CoP

				- Limites vitesses/accélérations de la base

				- Respecter la cinématique : C-B

			\subsubsection{Problème quadratique résultant}
		
				- Ecrire le problème résultant

		\subsection{Lorsque le robot bascule sur deux roues}
			\subsubsection{Formulation des objectifs}
		
				- Minimiser l'angle

				- Minimiser la vitesse angulaire

				- Stabilité numérique

			\subsubsection{Formulation des contraintes}

				- Contrainte sur l'angle > 0

				- Respecter la cinématique : C-B

				- Limites vitesses/accélérations de la base

			\subsubsection{Problème quadratique résultant}
		
				- Ecrire le problème résultant

		\subsection{Gestion de la transition entre les deux états}
			\subsubsection{Formulation des objectifs}

				- Minimiser la vitesse

				- Robustesse (CoP)

				- Stabilité numérique (jerk)

			\subsubsection{Formulation des contraintes}

				- Respecter la dynamique : CoP

				- Limites vitesses/accélérations de la base

				- Respecter la cinématique : C-B

			\subsubsection{Problème quadratique résultant}
		
				- Ecrire le problème résultant

	\section{Gestion des deux modèles dynamiques exclusifs}
		\subsection{Choix d'un superviseur et conséquences}

			- Problème de transitions entre les controlleurs

			- Il faut un superviseur qui gère les différents états

			- Parler de l'estimateur d'impact

			- Limitations due au superviseur : Détection tardive / inadéquate / Choix non optimal / Oscillations

			- Avantages : Gérer de manière simple différents modèles dynamiques

		\subsection{Fonctionnement du superviseur}

			- Expliquer le fonctionnement du superviseur et des différents états

		\subsection{Fonctionnement de l'estimateur d'impact}

			- Détailler le fonctionnement de l'estimateur d'impact

	\section{Vers une modélisation unifiée des deux dynamiques}
		\subsection{Problème de complémentarité linéaire}

			- Considérer uniquement un problème de basculement dans une direction

			- dire que par la suite, se limiter à ce cas permet de gérer tout les cas, en faisant quelques hyothèses

			- Enoncer la dynamique de complémentarité

			- Problème : Il y a $2^n$ états possibles linéaires à la dynamique.

		\subsection{Méthodes de résolution}
				- Considérer que lorsque le robot ne bascle pas, la commande ne le fera pas basculer. On se retrouver dans le cas du premier programme d'optimisation uniquement

				- Si un baculement est mesure, faire un apriori qu'il n'y aura pas de rebond possible. Ainsi, il n'y a qu'une variable à choisir : le temps d'impact.

				- On se retrouve avec un problème non-linéaire, qui devient linéaire en choisissant l'état de cette variable.

				- Il y a n choix possibles.

				- On peut résoudre n QP et choisir le plus optimal.

				- Ou alors on peut résoudre 3 QP et faire converger l'état de la variable.

				- Présenter le problème d'optimisation unifié