\chapter{Commande par modèle prédictif}

	~

	\section{Principe}
	\section{Outil mathématique et contraintes associées}
		\subsection{Problématique}
		\subsection{Principe de la programmation quadratique}
		\subsection{Contraintes sur la formulation du problème d'optimisation}
	\section{Formulation des problèmes d'optimisations}
		\subsection{Introduction}
		\subsection{Lorsque les trois roues sont en contact avec le sol}
			\subsubsection{Formulation des objectifs}
			\subsubsection{Formulation des contraintes}
			\subsubsection{Problème quadratique résultant}
		\subsection{Lorsque le robot bascule sur deux roues}
			\subsubsection{Formulation des objectifs}
			\subsubsection{Formulation des contraintes}
			\subsubsection{Problème quadratique résultant}
		\subsection{Gestion de la transition entre les deux états}
			\subsubsection{Problématique associée}
			\subsubsection{Formulation des objectifs}
			\subsubsection{Formulation des contraintes}
			\subsubsection{Problème quadratique résultant}
			
	~
			
	\section{Gestion des deux modèles dynamiques exclusifs}
		\subsection{Choix d'un superviseur et conséquences}
		\subsection{Fonctionnement du superviseur}
		\subsection{Fonctionnement de l'estimateur d'impact}
	\section{Vers une modélisation unifiée des deux dynamiques}
		\subsection{Problème de complémentarité linéaire}
		\subsection{Méthodes de résolution}
			\subsubsection{Programmation quadratique avec contraintes non-linéaires}
			\subsubsection{Linéarisation par \textit{a priori}}
			\subsubsection{Conclusion}