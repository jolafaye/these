\section{Modélisation dynamique}
\subsection{Problématiques supplémentaires}

Parler du sous-actionnement
Parler du changement de dynamique (ajout d'une variable)
Parler des impacts

\subsection{Équations de la dynamique}

Présentation des nouvelles équations de la dynamique

\subsection{Linéarisation et approximations}

Considérer un approximation aux petits angles
Considérer une hauteur constante du robot dans le repère robot
Ne pas prendre en compte les moments des corps

\section{Commande prédictive}
\subsection{Modélisation de la dynamique future}

Définir les variables
Calculer la dynamique de l'angle.

\subsection{Formulation des objectifs}

Présenter les objectifs

\subsection{Formulation des contraintes}

Présenter les contraintes linéaires

\section{Gestion des deux modèles dynamiques exclusifs}
\subsection{Choix d'un superviseur et conséquences}

Problématique de l'impact et de la phase d'atterrissage.
Présentation du superviseur et de l'estimateur.

\subsection{Fonctionnement du superviseur}

Présenter la fsm

\subsection{Fonctionnement de l'estimateur d'impact}

Détailler les équations de calcul de la vitesse et du temps d'impact.\\
Discuter de la validité du modèle.

\section{Résultats et expérimentations}
\subsection{Protocole expérimental}

Différents push avec des balles.\\
Importance du point d'impact (déplacement base / basculement).\\
Parler de l'observation de l'angle de basculement.

\subsection{Expérimentations}

Analyser les expériences

\subsection{Limites physiques et algorithmiques}

Parler des limites en accélération/vitesse des roues.\\
Parler du retard initial lié à l'observation et au superviseur.\\
Parler des moments des bras


\section{Vers une modélisation unifiée des deux dynamiques}
\subsection{Problème de complémentarité linéaire}

Présenter le problème complet.\\
Problématique de la complémentarité linéaire.\\
Nécessité d'un algorithme non-linéaire ou d'apriori.\\
contrainte du temps de calcul.

\subsection{Méthodes de résolution}
\subsubsection{Programmation quadratique avec contraintes non-linéaire}

Présentation des sqp

\subsubsection{Linéarisation par \textit{a priori}}

Choix qui parrait le plus pertinent (non rebond, variation non linéaire du temps d'impact)

\subsubsection{Conclusion}

Avantages et inconvénients d'une modélisation unifiée.\\
Sensibilité du contrôleur à l'observation de l'angle.