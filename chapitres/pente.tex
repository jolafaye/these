\section{Modélisation dynamique}
\subsection{Problématique supplémentaire}

Parler des changement de pente lors du déplacement, et de la pente initiale.
Variation du vecteur gravité

\subsection{Équations de la dynamique}
\subsection{Lorsque les trois roues sont au sol}

Détailler les équations.

\subsection{Lorsque deux roues sont au sol}

Détailler les équations

\subsection{Prédiction du vecteur gravité}

Considérer deux optiques : valeur constante et extrapolation\\
Avantages et inconvénients des deux.

\section{Observabilité}
\subsection{Problématique associée}

Insuffisance des capteurs pour déterminer chaque angle\\
Besoin d'un observateur avec a priori (non observable directement)

\subsection{Observation simultanée de l'angle de la pente et de basculement}

Présentation des a priori.\\
Algorithme d'observation des deux angles

\subsection{Limitations}

Slow push\\
Observation pendant un déplacement (accélération parasite)\\
Modification de pente pendant un push\\
Bruit généré par la forme des roues

\section{Résultats et expérimentations}
\subsection{Protocole expérimental}

Montée de pente sous différents angles.\\
Push et pente

\subsection{Expérimentations}

Analyser les résultats

\subsection{Vers une meilleure observation des angles de pente et de basculement}

Utilisation possible d'autres capteurs.\\
Avantages et inconvénients.\\
Ajout de nouveaux capteurs