\documentclass[12pt]{report}

\input{../macro}


\usepackage[utf8]{inputenc}
\usepackage[T1]{fontenc} 
\usepackage{xspace}
\usepackage[frenchb]{babel}
\usepackage{times}
\usepackage{graphicx}
\usepackage[top=2in, bottom=1.5in, left=1in, right=1in]{geometry}
\usepackage{amsmath}

\graphicspath{{images/}}

%Add subsubsubsection%%%%%%%%%%%%%%%%%%
\addtocounter{tocdepth}{4}
\setcounter{secnumdepth}{4}
\makeatletter
\newcounter {subsubsubsection}[subsubsection]
\renewcommand\thesubsubsubsection{\thesubsubsection .\@alph\c@subsubsubsection}
\newcommand\subsubsubsection{\@startsection{subsubsubsection}{4}{\z@}%
                                     {-3.25ex\@plus -1ex \@minus -.2ex}%
                                     {1.5ex \@plus .2ex}%
                                     {\normalfont\normalsize\bfseries}}
\newcommand*\l@subsubsubsection{\@dottedtocline{3}{10.0em}{4.1em}}
\newcommand*{\subsubsubsectionmark}[1]{}
\makeatother
%%%%%%%%%%%%%%%%%%%%%%%%%%%%%%%%%%%%%%%

\setcounter{MaxMatrixCols}{20}

\bibliographystyle{plain}
\renewcommand\thechapter {\Roman{chapter}}

\renewcommand{\baselinestretch}{1.2}

\hyphenation{con-train-te}
\hyphenation{con-train-dre}
 
\begin{document}

Soit un point matériel $c$ se déplacant sur une dimension en suivant une dynamique d'ordre 3. 

Ce point est contraint à se déplacer dans l'intervalle $[-1, +1]$.

Ce point à pour objectif de suivre une trajectoire de référence $c^*$ définie sur l'horizon comme suis :
\eq{
	C^* = \mat{2 \\ -2 \\ 2 \\ -2 \\ \vdots}
}

On écrit une loi de commande prédictive pour ce système :
\eqa{
	&\lst{
		\min\limits_{\dddot{C}}\prt{O(\dddot{C})} \\
		-1 \leq C \leq 1
	} \\
	&O(\dddot{C})=\norm{C-C^*}^2
}
avec la relation dynamique :
\eq{
	C = U\dddot{C}
}
si l'on considère le point $c$ immobile à $t=0$ : $c=\dot{c}=\ddot{c}=0$.

Pour un nombre d'échantillons $n=5$:
\eq{
	U = \frac{T^3}{6} 
	\mat{
		1 & 0 & 0 & 0 & 0 \\
		4 & 1 & 0 & 0 & 0 \\
		7 & 4 & 1 & 0 & 0 \\
		19 & 7 & 4 & 1 & 0 \\
		37 & 19 & 7 & 4 & 1
	}
}
\eq{
	U^{-1} = \frac{6}{T^3}
	\mat{
		1 & 0 & 0 & 0 & 0 \\
		-4 & 1 & 0 & 0 & 0 \\
		9 & -4 & 1 & 0 & 0 \\
		-27 & 9 & -4 & 1 & 0 \\
		84 & -27 & 9 & -4 & 1
	}
}


On considère que la stabilité interne est assurée si les variables $\dddot{C}$ ne divergent pas.

On considère que la stabilité externe est assurée si les contraintes sont respectées (problème solvable) et si la valeur de la fonction objectif ne diverge pas.

Dans ce problème simple, la solution du problème de minimisation est toujours aux bornes des contraintes, où la fonction objectif est minimale :
\eq{
	O(\dddot{C})_{\min}=n \iff C_{opt}=\mat{1\\-1\\1\\-1\\1} 
}

La solution est directement :
\eqa{
	\dddot{C}_{opt} &= U^{-1}C_{opt}\\
	\dddot{C}_{opt} &= \frac{6}{T^3}\mat{
		1 & 0 & 0 & 0 & 0 \\
		-4 & 1 & 0 & 0 & 0 \\
		9 & -4 & 1 & 0 & 0 \\
		-27 & 9 & -4 & 1 & 0 \\
		84 & -27 & 9 & -4 & 1
	}
	\mat{1\\-1\\1\\-1\\1} \\
	\dddot{C}_{opt} &= \frac{6}{T^3}\mat{1\\-5\\14\\-41\\125}
}

Dans ce problème, la stabilité externe est assurée (solution respectant les contraintes, et fonction objectif minimale). Cependant, la stabilité interne ne l'est pas, car les $\dddot{c}_{opt}$ tendent vers l'infini.

En ajoutant un second objectif de minimisation du jerk $\dddot{C}$, l'importance relative de celui-ci par rapport au suivi de la trajectoire $c^*$ va faire en sorte que $\dddot{C}$ ne divergera pas :
\liste{
	\item Lorsque le jerk est faible, le suivi de trajectoire est prépondérant devant la minimisation du jerk. Le jerk grandit alors, et la valeur de l'objectif de suivi de trajectoire va diminuer vers son minimum.
	\item Lorsque le jerk devient grand, la minimisation du jerk est prépondérant devant le suivi de trajectoire. Le jerk va donc diminuer, et la valeur de l'objectif de suivi trajectoire va augmenter.
}

La valeur de l'objectif de suivi de trajectoire va donc soit suivre un cycle, soit se stabiliser. Le jerk quant à lui sera limité en fonction du choix du jeu de pondérations pour les deux objectifs.
\end{document}